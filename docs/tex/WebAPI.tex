\documentclass[11pt,a4paper]{scrartcl}

% NOTE: packages used
\usepackage[utf8x]{inputenc}
\usepackage{geometry}
\usepackage{graphicx}
\usepackage{amssymb}
\usepackage{hyperref}
\usepackage{epstopdf}
\usepackage{color}
\usepackage{xcolor}
\usepackage{xcolor, listings}
\usepackage{caption}
\usepackage{minted}

% NOTE: declarations and definitions document wide:
\DeclareCaptionFont{white}{\color{white}}
\DeclareCaptionFormat{listing}{\colorbox{gray}{\parbox{\textwidth}{#1#2#3}}}
\captionsetup[lstlisting]{format=listing,labelfont=white,textfont=white}
\DeclareGraphicsRule{.tif}{png}{.png}{`convert #1 `dirname #1`/`basename #1 .tif`.png}

\setlength{\parskip}{1.2ex}
\setlength{\parindent}{2em}

% dark green:
\definecolor{green2}{HTML}{347235}

\title{ServeD WebAPI}
\author{\href{mailto:dmilith@verknowsys.com}{Daniel (dmilith) Dettlaff}}
\subtitle{version 0.6.10}
\date{2012-12-14}


% NOTE: document content:
\begin{document}

\maketitle

\section{HTTP JSON API Calls}\label{sec:apicalls}
  All calls are by default using HTTP POST method.
  An invalid request will return "status" code "1" error.

  \begin{itemize}

    \item \#000:
      \begin{minted}{bash}
    /Authorize
      \end{minted}
      \begin{minted}{scala}
      @description checks user key and creates svdauth cookie.
      @returns nothing.
      @params Authorize=userAuthorizationKey
      @format json: {message: information_string}
      \end{minted}

    \item \#001:
      \begin{minted}{bash}
    /GetUserProcesses
      \end{minted}
      \begin{minted}{scala}
      @description gets process information from all services and
        applications running by user (owner of web panel).
      @returns nothing or one of error codes: KDERR, NOPCS as
        message in case of critical failure.
      @params no additional params.
      @format json: {message: information_string}
      \end{minted}

    \item \#002:
      \begin{minted}{bash}
    /RegisterDomain
      \end{minted}
      \begin{minted}{scala}
      @description registers user domain.
      @returns nothing
      @params RegisterDomain=domainNameBoundToServer
      @format json: {message: information_string}
      \end{minted}

    \item \#003:
      \begin{minted}{bash}
    /GetRegisteredDomains
      \end{minted}
      \begin{minted}{scala}
      @description lists domains registered by user.
      @returns list of domains as content.
      @params no additional params.
      @format json: {message: information_string, content: [domain1, ...]}
      \end{minted}

    \item \#004:
      \begin{minted}{bash}
    /GetStoredServices
      \end{minted}
      \begin{minted}{scala}
      @description lists services stored by user.
      @returns list of stored services as content.
      @params no additional params.
      @format json: {message: information_string, content: [Svc1, ...]}
      \end{minted}

    \item \#005:
      \begin{minted}{bash}
    /TerminateServices
      \end{minted}
      \begin{minted}{scala}
      @description terminates all user spawned services.
      @returns nothing.
      @params no additional params.
      @format json: {message: information_string}
      \end{minted}

    \item \#006:
      \begin{minted}{bash}
    /StoreServices
      \end{minted}
      \begin{minted}{scala}
      @description stores user spawned services. it will cause
        autostart of those services on each WebPanel start.
      @returns nothing.
      @params no additional params.
      @format json: {message: information_string}
      \end{minted}

    \item \#007:
      \begin{minted}{bash}
    /SpawnService
      \end{minted}
      \begin{minted}{scala}
      @description spawns given service.
      @returns nothing.
      @params SpawnService=ServiceName
      @format json: {message: information_string}
      \end{minted}

    \item \#008:
      \begin{minted}{bash}
    /TerminateService
      \end{minted}
      \begin{minted}{scala}
      @description terminates service given as a param.
      @returns nothing.
      @params TerminateService=ServiceName
      @format json: {message: information_string}
      \end{minted}

    \item \#009:
      \begin{minted}{bash}
    /ShowAvailableServices
      \end{minted}
      \begin{minted}{scala}
      @description shows all available services defined
        to be used by user.
      @returns list of available services as content array.
      @params no additional params.
      @format json: {message: information_string, content:[Svc1, ...] }
      \end{minted}

    \item \#010:
      \begin{minted}{bash}
    /SpawnServices
      \end{minted}
      \begin{minted}{scala}
      @description spawns all stored services at once.
      @returns status code 0 when service is started,
        error otherwise.
      @params no additional params.
      @format json: {message: information_string}
      \end{minted}

    \item \#011:
      \begin{minted}{bash}
    /GetServiceStatus
      \end{minted}
      \begin{minted}{scala}
      @description checks service status.
      @returns status code 0 when service is started with additional service uptime information,
        error otherwise.
      @params GetServiceStatus=ServiceName
      @format json: {message: information_string}
      \end{minted}

    \item \#012:
      \begin{minted}{bash}
    /GetServicePort
      \end{minted}
      \begin{minted}{scala}
      @description retrieves service port on which given service is listening at.
      @returns port list as content
      @params GetServicePort=ServiceName
      @format json: {message: information_string, content: [1025, 1026, ...]}
      \end{minted}

    \item \#013:
      \begin{minted}{bash}
    /CloneIgniterForUser
      \end{minted}
      \begin{minted}{scala}
      @description creates a copy of existing service igniter for user.
      @returns nothing.
      @params IgniterName=originalServiceIgniter, UserIgniterName=userServiceName
      @format json: {message: information_string}
      \end{minted}

    \item \#014:
      \begin{minted}{bash}
    /RegisterAccount
      \end{minted}
      \begin{minted}{scala}
      @description registers unique user name to prevent using it by others.
      @returns nothing.
      @params RegisterAccount=nameOfAccount
      @format json: {message: information_string}
      \end{minted}

    \item \#015:
      \begin{minted}{bash}
    /CreateFileWatch
      \end{minted}
      \begin{minted}{scala}
      @description registers file event monitor on file or directory. service is starting in case of sufficient flag trigger.
      @params CreateFileWatch=absoluteFilePath&ServiceName=nameOfService&Flags=3
      @format json: {message: information_string}
      \end{minted}

    \item \#016:
      \begin{minted}{bash}
    /DestroyFileWatch
      \end{minted}
      \begin{minted}{scala}
      @description destroys file watches on given file (all!)
      @params DestroyFileWatch=absoluteFilePath
      @format json: {message: information_string}
      \end{minted}

    \item \#017:
      \begin{minted}{bash}
    /GetAccountPriviledges
      \end{minted}
      \begin{minted}{scala}
      @description send query to system manager for user priviledges info.
      @params nothing.
      @format json: {message: information_string, content: [priviledges_objects]}
      \end{minted}


    \item \#018:
      \begin{minted}{bash}
    /RestartAccountManager
      \end{minted}
      \begin{minted}{scala}
      @description stops Account Manager.
      @params nothing.
      @format json: {message: information_string}
      \end{minted}

    \item \#019:
      \begin{minted}{bash}
    /GetUserPorts
      \end{minted}
      \begin{minted}{scala}
      @description returns list of ports registered by user..
      @params nothing.
      @format json: {message: information_string, content: [list_of_ports]}
      \end{minted}

    \item \#020:
      \begin{minted}{bash}
    /RemoveAllReservedPorts
      \end{minted}
      \begin{minted}{scala}
      @description removes all ports registered by user.
      @params nothing.
      @format json: {message: information_string}
      \end{minted}

    \item \#021:
      \begin{minted}{bash}
    /RegisterUserPort
      \end{minted}
      \begin{minted}{scala}
      @description registers random free user port.
      @params nothing.
      @format json: {message: information_string, contents: [port]}
      \end{minted}


  \end{itemize}


\section{HTTP JSON API Errors}\label{sec:apierrors}
  All errors should have only two fields with JSON object respond: "status" and "message".

  \begin{itemize}
    \item "message" contains detailed information about error.
    \item "status" codes:
      \begin{description}
        \item{0} - API Request executed successfully.
        \item{1} - Invalid API Request.
        \item{2} - Error executing API Request.
        \item{3} - API Request Timeout.
        \item{4} - API Request Crashed.
        \item{5} - API Access Denied.
        \item{-2} - Native error: NOPCS.
        \item{-1} - Native error: KDERR.
      \end{description}

  \end{itemize}


\section{Scala code usage examples}\label{sec:scalausage}
  Scala code examples to be used in own code.

  \begin{minted}[linenos,
               numbersep=4pt,
               gobble=2,
               frame=lines,
               framesep=2mm]{scala}
  /** API call #001  */
    case req @ POST(Path(Seg("GetUserProcesses" :: Nil))) =>
      SvdWebAPI.apiRespond(webManager ? System.GetUserProcesses(account.uid))

  /** API call #002  */
    case req @ POST(Path(Seg("RegisterDomain" :: domain :: Nil))) =>
      SvdWebAPI.apiRespond(webManager ? System.RegisterDomain(domain))

  \end{minted}


\end{document}
