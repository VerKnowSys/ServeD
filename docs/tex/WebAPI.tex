\documentclass[11pt,a4paper]{scrartcl}

% NOTE: packages used
\usepackage[utf8x]{inputenc}
\usepackage{geometry}
\usepackage{graphicx}
\usepackage{amssymb}
\usepackage{hyperref}
\usepackage{epstopdf}
\usepackage{color}
\usepackage{xcolor}
\usepackage{xcolor, listings}
\usepackage{caption}
\usepackage{minted}

% NOTE: declarations and definitions document wide:
\DeclareCaptionFont{white}{\color{white}}
\DeclareCaptionFormat{listing}{\colorbox{gray}{\parbox{\textwidth}{#1#2#3}}}
\captionsetup[lstlisting]{format=listing,labelfont=white,textfont=white}
\DeclareGraphicsRule{.tif}{png}{.png}{`convert #1 `dirname #1`/`basename #1 .tif`.png}

\setlength{\parskip}{1.2ex}
\setlength{\parindent}{2em}

% dark green:
\definecolor{green2}{HTML}{347235}

\title{ServeD WebAPI}
\subtitle{version 0.1.3}
\author{\href{mailto:dmilith@verknowsys.com}{Daniel (dmilith) Dettlaff}}
\date{2012-12-14}

\lstloadlanguages{Ruby}
\lstset{
  language=Ruby,
  morecomment=[s][\color{blue}]{\:}{\ },
  morecomment=[s][\color{magenta}]{\=\>}{\ },
  basicstyle=\tiny,
  tabsize=4,
  breaklines=true,
  breakatwhitespace=true,
  numberstyle=\tiny,
  firstnumber=1,
  basicstyle=\ttfamily\color{black},
  commentstyle=\ttfamily\color{gray},
  keywordstyle=\ttfamily\color{red},
  stringstyle=\color{green2},
  numberstyle=\ttfamily\color{magenta}
}

% NOTE: document content:
\begin{document}

\maketitle

\section{HTTP JSON API Calls}\label{sec:apicalls}
  All calls are by default using HTTP POST method.
  An invalid request will return "status" code "1" error.

  \begin{itemize}

    \item \#001 - /GetUserProcesses - Gets process information from all services and applications running by user (owner of web panel).
      \begin{itemize}
        \item params: no additional params.
        \item return type: JSON.
        \item errors: \{message: "information"\} where MESSAGE is one of error codes: KDERR, NOPCS.
        \item format: \{message: "information", content: [\{properties, …\}, …]\}
      \end{itemize}

    \item \#002 - /RegisterDomain - Registers user domain.
      \begin{itemize}
        \item params: Accepts additional http param as domain, or takes \"RegisterDomain\" key from http body data array (when no additional http param is provided).
        \item return type: JSON
        \item format: \{message: "information", content: [\{properties, …\}, …]\}
      \end{itemize}

    \item \#003 - /RegisteredDomains - Lists domains registered by user.
      \begin{itemize}
        \item params: no additional params.
        \item return type: JSON
        \item format: \{message: "information", content: [domain1, domain2, …]\}
      \end{itemize}

  \end{itemize}


\section{HTTP JSON API Errors}\label{sec:apierrors}
  All errors should have only two fields with JSON object respond: "status" and "message".

  \begin{itemize}
    \item "message" contains detailed information about error.
    \item "status" codes:
      \begin{description}
        \item{0} - API Request executed successfully.
        \item{1} - Invalid API Request.
        \item{2} - Error executing API Request.
        \item{3} - API Request Timeout.
        \item{4} - API Request Crashed.
        \item{-2} - Native error: NOPCS.
        \item{-1} - Native error: KDERR.
      \end{description}

  \end{itemize}


\section{Scala code usage examples}\label{sec:scalausage}
  Scala code examples to be used in own code.

  \begin{minted}[linenos,
               numbersep=4pt,
               gobble=2,
               frame=lines,
               framesep=2mm]{scala}
  /** API call #001  */
    case req @ POST(Path(Seg("GetUserProcesses" :: Nil))) =>
      SvdWebAPI.apiRespond(webManager ? System.GetUserProcesses(account.uid))

  /** API call #002  */
    case req @ POST(Path(Seg("RegisterDomain" :: domain :: Nil))) =>
      SvdWebAPI.apiRespond(webManager ? System.RegisterDomain(domain))

  \end{minted}


\end{document}
