\documentclass[11pt,a4paper]{scrartcl}

% NOTE: packages used
\usepackage[utf8x]{inputenc}
\usepackage{polski}
\usepackage{geometry}
% \usepackage{graphicx}
\usepackage{amssymb}
\usepackage{hyperref}
\usepackage{epstopdf}
\usepackage{color}
% \usepackage{xcolor}
% \usepackage{xcolor, listings}
% \usepackage{caption}
\usepackage{minted}

% NOTE: declarations and definitions document wide:
% \DeclareCaptionFont{white}{\color{white}}
% \DeclareCaptionFormat{listing}{\colorbox{gray}{\parbox{\textwidth}{#1#2#3}}}
% \captionsetup[lstlisting]{format=listing,labelfont=white,textfont=white}
% \DeclareGraphicsRule{.tif}{png}{.png}{`convert #1 `dirname #1`/`basename #1 .tif`.png}

% \setlength{\parskip}{1.2ex}
% \setlength{\parindent}{2em}

% % dark green:
% \definecolor{green2}{HTML}{347235}

\title{}
\subtitle{version 0.1.0}
\author{\href{mailto:dmilith@verknowsys.com}{Daniel (dmilith) Dettlaff}}
\date{2012-12-17}

% \lstloadlanguages{Ruby}
% \lstset{
%   language=Ruby,
%   morecomment=[s][\color{blue}]{\:}{\ },
%   morecomment=[s][\color{magenta}]{\=\>}{\ },
%   basicstyle=\tiny,
%   tabsize=4,
%   breaklines=true,
%   breakatwhitespace=true,
%   numberstyle=\tiny,
%   firstnumber=1,
%   basicstyle=\ttfamily\color{black},
%   commentstyle=\ttfamily\color{gray},
%   keywordstyle=\ttfamily\color{red},
%   stringstyle=\color{green2},
%   numberstyle=\ttfamily\color{magenta}
% }

% NOTE: document content:
\begin{document}

\maketitle

% \section{Ankieta}\label{sec:ankieta}
Warunki przystąpienia do realizacji projektu Serwisu Internetowego.

\begin{itemize}

  \item Projekt musi być kompletny, przez co implikuje się, że:
  \begin{enumerate}
    \item Znany jest wygląd \underline{każdej} zdefiniowanej podstrony i określony jest sposób zachowania dla każdej z nich. np. Jeśli zdjęcie ma otwierać element, to musi być określone jak ma wyglądać widok tego podglądu.
    \item Zdefiniowano ile podstron ma strona - jeśli jest ich określona (stała) ilość - lub zdefiniowano w jaki sposób ma się zachowywać strona przy ich dodawaniu/ usuwaniu.
    \item Zdefiniowano (proponowane przez klienta) zachowanie mechanizmów \underline{każdej} z podstron.
  \end{enumerate}

\end{itemize}

\end{document}
